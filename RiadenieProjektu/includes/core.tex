% !TEX root = ../example_paper.tex
\section{Predstavenie Riešiteľského Kolektívu}

Tu uveďte informácie o členoch vášho tímu, vrátane ich rolí a zodpovedností.

\section{Dohodnuté Metódy Práce, Komunikácie a Koordinácie Projektu}
V našom tíme sme dohodli nasledujúce rozdelenie úloh a zodpovedností:

\begin{itemize}
    \item \textbf{3D CAD Modelovanie:} Táto časť projektu bola rozdelená medzi Matúša a Jakuba. Sú zodpovední za vytvorenie 3D modelov potrebných pre náš projekt.
    
    \item \textbf{Návrh PCB a Veci Spojené s Hardvérom:} Maroš a Erik sa zaoberajú návrhom PCB a ostatnými aspektmi spojenými s hardvérom nášho projektu.
    
    \item \textbf{Softvérový Vývoj:} Tím Edo, Timo a Filip majú na starosti softvérovú časť projektu. Ich zodpovednosťou je vývoj softvéru, vrátane návrhu a implementácie regulátorov, komunikácie medzi myšou a programom a algoritmov.
\end{itemize}

Toto rozdelenie úloh nám umožňuje efektívne pracovať na každej časti projektu a zabezpečiť, že každý člen tímu má jasne definované zodpovednosti. V prípade potreby sa samozrejme môžu ľudia podieľať aj na iných aspektoch projektu z dôvodu urychlenia práce alebo lepšieho výsledku. 

Pre rýchlu komunikáciu a diskusiu využívame platformu Discord, na správu kódu a verzionovanie slúži GitHub. Pravidelné týždňové stretnutia nám umožňujú zhodnotiť pokrok a riešiť prípadné otázky alebo problémy.

\section{Plán Projektu}

\subsection{Týždeň 1}
\begin{itemize}
    \item \textbf{Stretnutie:} 20/02/2024
    \item \textbf{Cieľ: } blabla
    \item \textbf{Zoznam Úloh:}
\end{itemize}

\begin{tabular}{|l|l|}
    \hline
    \textbf{Člen} & \textbf{Úloha} \\
    \hline
    Eduard Zelenay & Zoznamovanie sa s platformou esp-idf \\
    Erik Môcik & Rešerš potrebných senzorov \\
    Filip Lobpreis & Vyhľadanie vhodných algoritmov a logiky \\
    Jakub Močarník & Vyjasnenie priradenia k projektu\\
    Maroš Kocúr & Navrhnutie a vytvorenie PCB v programe KiCAD \\
    Matúš Machata & Diskusia ohľadom výberu motorov a PCB dosky, prvotné námety na riešenie HW \\
    Timotej Polc & Konzultácia spojenia hardvéru so softvérom \\
    \hline
\end{tabular}
\subsection{Týždeň 2}
\begin{itemize}
    \item \textbf{Stretnutie:} 27/02/2024
    \item \textbf{Zoznam Úloh:}
\end{itemize}

\begin{tabular}{|l|l|}
    \hline
    \textbf{Člen} & \textbf{Úloha} \\
    \hline
    Eduard Zelenay & Študovanie datasheetu ku ToF senzorom \\
    Erik Môcik & Rešerš potrebných senzorov \\
    Filip Lobpreis & Vyhľadanie vhodných algoritmov a logiky \\
    Jakub Močarník & Vyjasnenie priradenia k projektu \\
    Maroš Kocúr & Navrhnutie a vytvorenie PCB v programe KiCAD \\
    Matúš Machata & Príprava informácií a zdrojových materiálov (3D model motora, model PCB dosky) \\
    Timotej Polc & Konzultácia spojenia hardvéru so softvérom \\
    \hline
\end{tabular}
\newpage
\subsection{Týždeň 3}
\begin{itemize}
    \item \textbf{Stretnutie:} 05/03/2024
    \item \textbf{Zoznam Úloh:}
\end{itemize}

\begin{tabular}{|l|l|}
    \hline
    \textbf{Člen} & \textbf{Úloha} \\
    \hline
    Eduard Zelenay & Tvorba knižnice pre senzory \\
    Erik Môcik & Rešerš potrebných senzorov \\
    Filip Lobpreis & Vyhľadanie vhodných algoritmov a logiky \\
    Jakub Močarník & Vyjasnenie priradenia k projektu \\
    Maroš Kocúr & Navrhnutie a vytvorenie PCB v programe KiCAD \\
    Matúš Machata & Prvotný návrh kolies a prevodovky, diskusia návrhov \\
    Timotej Polc & Konzultácia spojenia hardvéru so softvérom \\
    \hline
\end{tabular}
\subsection{Týždeň 4}
\begin{itemize}
    \item \textbf{Stretnutie:} 12/03/2024
    \item \textbf{Zoznam Úloh:}
\end{itemize}

\begin{tabular}{|l|l|}
    \hline
    \textbf{Člen} & \textbf{Úloha} \\
    \hline
    Eduard Zelenay & Tvorba knižnice pre senzory \\
    Erik Môcik & Rešerš potrebných senzorov \\
    Filip Lobpreis & Vyhľadanie vhodných algoritmov a logiky \\
    Jakub Močarník & Zaradenie k téme, priradenie CAD časti, návrh úchytov, kolies a prevodov spolu s Matúšom Machatom. \\
    Maroš Kocúr & Navrhnutie a vytvorenie PCB v programe KiCAD \\
    Matúš Machata & Prerábanie 3D modelu prevodovky \\
    Timotej Polc & Konzultácia spojenia hardvéru so softvérom \\
    \hline
\end{tabular}
\subsection{Týždeň 5}
\begin{itemize}
    \item \textbf{Stretnutie:} 19/03/2024
    \item \textbf{Zoznam Úloh:}
\end{itemize}

\begin{tabular}{|l|l|}
    \hline
    \textbf{Člen} & \textbf{Úloha} \\
    \hline
    Eduard Zelenay & Tvorba knižnice pre senzory \\
    Erik Môcik & Rešerš potrebných senzorov \\
    Filip Lobpreis & Vyhľadanie vhodných algoritmov a logiky \\
    Jakub Močarník & Prvé návrhy, prvotný návrh úchytov \\
    Maroš Kocúr & Navrhnutie a vytvorenie PCB v programe KiCAD \\
    Matúš Machata & Rešerš vhodných skrutiek, matiek a ostatných potrebných komponentov \\
    Timotej Polc & Konzultácia spojenia hardvéru so softvérom \\
    \hline
\end{tabular}
\subsection{Týždeň 6}
\begin{itemize}
    \item \textbf{Stretnutie:} 26/03/2024
    \item \textbf{Zoznam Úloh:}
\end{itemize}

\begin{tabular}{|l|l|}
    \hline
    \textbf{Člen} & \textbf{Úloha} \\
    \hline
    Eduard Zelenay & Debugovanie, testovanie senzoru \\
    Erik Môcik & Rešerš potrebných senzorov \\
    Filip Lobpreis & Vyhľadanie vhodných algoritmov a logiky \\
    Jakub Močarník & Prvý návrh s motormi pre našu aplikáciu, nasledovala tlač a pripomienkovanie nedostatkov \\
    Maroš Kocúr & Navrhnutie a vytvorenie PCB v programe KiCAD \\
    Matúš Machata & Prvá tlač 3D návrhu, modelovanie nového návrh z dôvodu menších nezrovnalostí \\
    Timotej Polc & Konzultácia spojenia hardvéru so softvérom \\
    \hline
\end{tabular}
\subsection{Týždeň 7}
\begin{itemize}
    \item \textbf{Stretnutie:} 02/04/2024
    \item \textbf{Zoznam Úloh:}
\end{itemize}

\begin{tabular}{|l|l|}
    \hline
    \textbf{Člen} & \textbf{Úloha} \\
    \hline
    Eduard Zelenay & Implementácia 4 senzorov na reálnom zariadení \\
    Erik Môcik & Rešerš potrebných senzorov \\
    Filip Lobpreis & Vyhľadanie vhodných algoritmov a logiky \\
    Jakub Močarník & Úprava komponentov na základe zistení z predošlej iterácie a ich nová tlač \\
    Maroš Kocúr & Navrhnutie a vytvorenie PCB v programe KiCAD \\
    Matúš Machata & Tlač ďalšej iterácie návrhu, čistenie a úprava výtlačkov \\
    Timotej Polc & Konzultácia spojenia hardvéru so softvérom \\
    \hline
\end{tabular}
\subsection{Týždeň 8}
\begin{itemize}
    \item \textbf{Stretnutie:} 09/04/2024
    \item Dlho očakávaný príchod motorov Faulhaber.
    \item \textbf{Zoznam Úloh:}
\end{itemize}

\begin{table}[H]
    \centering
    \begin{tabular}{|l|p{0.6\linewidth}|}
        \hline
        \textbf{Člen} & \textbf{Úloha} \\
        \hline
        Eduard Zelenay & Code review, debugovanie gyroskopu \\
        Erik Môcik & Rešerš potrebných senzorov \\
        Filip Lobpreis & Vyhľadanie vhodných algoritmov a logiky \\
        Jakub Močarník & Dopasovanie rozmerov komponentov na lepšie tolerancie na základe požadovaných vlastností \\
        Maroš Kocúr & Navrhnutie a vytvorenie PCB v programe KiCAD \\
        Matúš Machata & Nová tlač kolies a prevodovky z dôvodu nedostatočnej presnosti tlačiarne, diskusia ohľadom nahradenia prevodovky vyrobenou vstrekolisom \\
        Timotej Polc & Konzultácia spojenia hardvéru so softvérom \\
        \hline
    \end{tabular}
\end{table}
\subsection{Týždeň 9}
\begin{itemize}
    \item \textbf{Stretnutie:} 16/04/2024
    \item \textbf{Zoznam Úloh:}
\end{itemize}

\begin{tabular}{|l|l|}
    \hline
    \textbf{Člen} & \textbf{Úloha} \\
    \hline
    Eduard Zelenay & Testovanie inej open-source knižnice \\
    Erik Môcik & Rešerš potrebných senzorov \\
    Filip Lobpreis & Vyhľadanie vhodných algoritmov a logiky \\
    Jakub Močarník & Pri tlači nám vznikali stále malé nepresnosti aj keď sa bral ohľad na smer tlače a aj veľkosť trisky, 
iterovanie viacero verzií naraz aby sa čo najviac okresali čas potrebný na konečný kus\\
    Maroš Kocúr & Navrhnutie a vytvorenie PCB v programe KiCAD \\
    Matúš Machata & Montovanie kolies a prevodovky pre prípravu montáže na PCB \\
    Timotej Polc & Konzultácia spojenia hardvéru so softvérom \\
    \hline
\end{tabular}
\subsection{Týždeň 10}
\begin{itemize}
    \item \textbf{Stretnutie:} 23/03/2024
    \item \textbf{Zoznam Úloh:}
\end{itemize}

\begin{tabular}{|l|l|}
    \hline
    \textbf{Člen} & \textbf{Úloha} \\
    \hline
    Eduard Zelenay & Code review odometrie a nekóderov \\
    Erik Môcik & Vytvorenie UDP servera v Pythone \\
    Filip Lobpreis & Vyhľadanie vhodných algoritmov a logiky \\
    Jakub Močarník & Pridanie úchytov na batériu, sprofilovanie komponentov \\
    Maroš Kocúr & Navrhnutie a vytvorenie PCB v programe KiCAD \\
    Matúš Machata & Finálne obrábanie výtlačkov pre SW testovanie \\
    Timotej Polc & Konzultácia spojenia hardvéru so softvérom \\
    \hline
\end{tabular}
\subsection{Týždeň 11}
\begin{itemize}
    \item \textbf{Stretnutie:} 30/04/2024
    \item \textbf{Zoznam Úloh:}
\end{itemize}

\begin{tabular}{|l|l|}
    \hline
    \textbf{Člen} & \textbf{Úloha} \\
    \hline
    Eduard Zelenay & Ladenie regulátora \\
    Erik Môcik & Grafické zobrazenie dát z enkóderov, ToF senzorov a gyroskopu pre jednoduchšie debugovanie \\
    Filip Lobpreis & Vyhľadanie vhodných algoritmov a logiky \\
    Jakub Močarník & Záverečné úpravy, presun zadných kolies kvôli nerovnomernosti tlače \\
    Maroš Kocúr & Navrhnutie a vytvorenie PCB v programe KiCAD \\
    Matúš Machata & Vytvorenie 3D modelu prevodovky \\
    Timotej Polc & Konzultácia spojenia hardvéru so softvérom \\
    \hline
\end{tabular}
\subsection{Týždeň 12}
\begin{itemize}
    \item \textbf{Stretnutie:} 07/05/2024
    \item \textbf{Zoznam Úloh:}
\end{itemize}

\begin{tabular}{|l|l|}
    \hline
    \textbf{Člen} & \textbf{Úloha} \\
    \hline
    Eduard Zelenay & Prerábanie UDP do C++ \\
    Erik Môcik & Spisovanie dokumentácie a riadenia projektu \\
    Filip Lobpreis & Vyhľadanie vhodných algoritmov a logiky \\
    Jakub Močarník & Posledné malé úpravy potrebné pre čo najlepší chod s finálnou iteráciou komponentov \\
    Maroš Kocúr & Navrhnutie a vytvorenie PCB v programe KiCAD \\
    Matúš Machata & Vytvorenie 3D modelu prevodovky \\
    Timotej Polc & Konzultácia spojenia hardvéru so softvérom \\
    \hline
\end{tabular}



\section{Podrobné Záznamy zo Stretnutí}

Uveďte podrobné záznamy zo stretnutí tímu vrátane rozhodnutí a kontroly rozhodnutí.
