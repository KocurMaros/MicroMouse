% !TEX root = ../example_paper.tex
\section{Dohodnuté Metódy Práce, Komunikácie a Koordinácie Projektu}
V našom tíme sme dohodli nasledujúce rozdelenie úloh a zodpovedností:

\begin{itemize}
    \item \textbf{3D CAD Modelovanie:} Táto časť projektu bola rozdelená medzi Matúša a Jakuba. Sú zodpovední za vytvorenie 3D modelov potrebných pre náš projekt.
    
    \item \textbf{Návrh DPS a Veci Spojené s Hardvérom:} Maroš a Erik sa zaoberajú návrhom DPS a ostatnými aspektmi spojenými s hardvérom nášho projektu.
    
    \item \textbf{Softvérový Vývoj:} Tím Edo, Timo a Filip majú na starosti softvérovú časť projektu. Ich zodpovednosťou je vývoj softvéru, vrátane návrhu a implementácie regulátorov, komunikácie medzi myšou a programom a algoritmov.
\end{itemize}

Toto rozdelenie úloh nám umožňuje efektívne pracovať na každej časti projektu a zabezpečiť, že každý člen tímu má jasne definované zodpovednosti. V prípade potreby sa samozrejme môžu ľudia podieľať aj na iných aspektoch projektu z dôvodu urychlenia práce alebo lepšieho výsledku. 

Pre rýchlu komunikáciu a diskusiu využívame platformu Discord, na správu kódu a verzionovanie slúži GitHub. Pravidelné týždňové stretnutia nám umožňujú zhodnotiť pokrok a riešiť prípadné otázky alebo problémy.

\section{Plán Projektu}

\subsection{Týždeň 1}
\begin{itemize}
    \item \textbf{Stretnutie:} 20/02/2024
    \item \textbf{Cieľ:} Výber MCU, senzorov a driverov pre motory a rozmerov myšky.}
    \item \textbf{Zoznam Úloh:}
\end{itemize}
\begin{table}[H]
    \centering
    \begin{tabular}{|l|p{0.6\linewidth}|}
        \hline
        \textbf{Člen} & \textbf{Úloha} \\
        \hline
        Eduard Zelenay & Zoznamovanie sa s platformou esp-idf \\
        \hline
        Erik Môcik & Rešerš potrebných senzorov \\
        \hline
        Filip Lobpreis & Zoznamovanie sa s platformou esp-idf \\
        \hline
        Jakub Močarník & Vyjasnenie priradenia k projektu\\
        \hline
        Maroš Kocúr & Navrhnutie a vytvorenie DPS v programe KiCAD \\
        \hline
        Matúš Machata & Vyjasnenie priradenia k projektu \\
        \hline
        Timotej Polc & Zoznamovanie sa s platformou esp-idf \\
        \hline
    \end{tabular}
\end{table}

\subsection{Týždeň 2}
Dizajn DPS a rozdelenie úloh podľa skúseností jednotlivých členov tímu.
\begin{itemize}
    \item \textbf{Stretnutie:} 27/02/2024
    \item \textbf{Cieľ: } Selekcia vhodných motorov, pozícia senzorov vzdialenosti a určenie bodu ťažiska. Debatovanie ohľadom nutnosti použitia kompasu pre spoľahlivejšie určovanie polohy.
    \item \textbf{Zoznam Úloh:}
\end{itemize}

\begin{table}[H]
    \centering
    \begin{tabular}{|l|p{0.6\linewidth}|}
        \hline
        \textbf{Člen} & \textbf{Úloha} \\
        \hline
        Eduard Zelenay & Študovanie datasheetu ku ToF senzorom \\
        \hline
        Erik Môcik & Rešerš vhodných miniatúrnych motorov a enkóderov \\
        \hline
        Filip Lobpreis & Detailné skúmanie hotových riešení MicroMouse  \\
        \hline
        Jakub Močarník & Vyjasnenie priradenia k projektu \\
        \hline
        Maroš Kocúr & Navrhnutie a vytvorenie DPS v programe KiCAD \\
        \hline
        Matúš Machata & Vyjasnenie priradenia k projektu \\
        \hline
        Timotej Polc & Rešerš vhodných miniatúrnych motorov a enkóderov \\
        \hline
    \end{tabular}
\end{table}
\newpage
\subsection{Týždeň 3}
Stretnutie s Pánom doktorom Jurajom Slačkom ohľadom dizajnu myšky.
\begin{itemize}
    \item \textbf{Stretnutie:} 05/03/2024
    \item \textbf{Cieľ: } Zvolenie efektívneho prevodového pomeru a polomeru kolies. Konzultácia návrhu bočného držiaku motorov a kolies.
    \item \textbf{Zoznam Úloh:}
\end{itemize}

\begin{table}[H]
    \centering
    \begin{tabular}{|l|p{0.6\linewidth}|}
        \hline
        \textbf{Člen} & \textbf{Úloha} \\
        \hline
        Erik Môcik & Výber motorov a enkóderov \\
        \hline
        Jakub Močarník & Návrh bočného držiaku motorov a kolies \\
        \hline
        Maroš Kocúr & Osádzanie a oživenie DPS \\
        \hline
        Matúš Machata & Prvotný návrh kolies a prevodovky \\
        \hline
    \end{tabular}
\end{table}

\subsection{Týždeň 4}
Výroba a osadenie DPS so základnými súčiastkami (pasívne, regulátory napätia, MCU)
\begin{itemize}
    \item \textbf{Stretnutie:} 12/03/2024
    \item \textbf{Cieľ: } Výber a objednanie vhodných senzorov na prototypizáciu.
    \item \textbf{Zoznam Úloh:}
\end{itemize}

\begin{table}[H]
    \centering
    \begin{tabular}{|l|p{0.6\linewidth}|}
        \hline
        \textbf{Člen} & \textbf{Úloha} \\
        \hline
        Eduard Zelenay & Naštudovanie použitého ToF senzoru \\
        \hline
        Erik Môcik & Objednanie senzorov(MPU 9250, VL53L1X) \\
        \hline
        Filip Lobpreis & Konzultácia logiky komunikácie SW s HW \\
        \hline
        Jakub Močarník & Tvorba ďaľšej iterácie modelov bočníc \\
        \hline
        Maroš Kocúr & Pridanie merania napätia batérie na DPS a obstaranie batérií \\
        \hline
        Matúš Machata & Prerábanie 3D modelu prevodovky \\
        \hline
        Timotej Polc & Konzultácia logiky komunikácie SW s HW \\
        \hline
    \end{tabular}
\end{table}

\subsection{Týždeň 5}
Konzultácia možnosti spolupráce pri objednaní súčiastok so školou prostredníctvom Pána docenta Andreja Babinca.
\begin{itemize}
    \item \textbf{Stretnutie:} 19/03/2024
    \item \textbf{Cieľ: } Umiestnenie gyroskopu a zvolenie vhodnej frekvencie obnovovania regulátora.
    \item \textbf{Zoznam Úloh:}
\end{itemize}

\begin{table}[H]
    \centering
    \begin{tabular}{|l|p{0.6\linewidth}|}
        \hline
        \textbf{Člen} & \textbf{Úloha} \\
        \hline
        Eduard Zelenay & Tvorba knižnice pre senzory \\
        \hline
        Filip Lobpreis & Vyhľadanie vhodných algoritmov a logiky \\
        \hline
        Jakub Močarník & Premodelovanie bočníc na iný motor \\
        \hline
        Maroš Kocúr & Programovanie štruktúry projektu v ESP-IDF \\
        \hline
        Matúš Machata & Rešerš vhodných skrutiek, matiek a ostatných potrebných komponentov \\
        \hline
        Timotej Polc & Programovanie štruktúry projektu v ESP-IDF \\
        \hline
    \end{tabular}
\end{table}

\subsection{Týždeň 6}
Dokončenie knižnice pre ToF senzor. Dodanie senzorov zakúpených školou. 
\begin{itemize}
    \item \textbf{Stretnutie:} 26/03/2024
    \item \textbf{Cieľ: } Vytvorenie knižnice pre gyroskop, regulátora pre motor a možnosti bezdrôtového posielania dát.
    \item \textbf{Zoznam Úloh:}
\end{itemize}

\begin{table}[H]
    \centering
    \begin{tabular}{|l|p{0.6\linewidth}|}
        \hline
        \textbf{Člen} & \textbf{Úloha} \\
        \hline
        Eduard Zelenay & Implementácia 4 senzorov na reálnom zariadení \\
        \hline
        Erik Môcik & Tvorba UDP klienta \\
        \hline
        Filip Lobpreis & Návrh regulátora na motor \\
        \hline
        Maroš Kocúr & Tvorba knižnice pre gyroskop \\
        \hline
        Timotej Polc & Vytvorenie odometrie \\
        \hline
    \end{tabular}
\end{table}

\subsection{Týždeň 7}
Motory boli nepoužiteľné, vzhľadom na ich veľkosť a nízke rozlíšenie enkóderov. Dodané motory od univerzity, ale bez enkóderov.
\begin{itemize}
    \item \textbf{Stretnutie:} 02/04/2024
    \item \textbf{Cieľ: } Zvolenie ďaľších krokov, kvôli blížiacemu sa termínu súťaže ISTROBOT. Reklamácia motorov.
    \item \textbf{Zoznam Úloh:} V tomto týždni neboli rozvrhnuté žiadne úlohy, kvôli dodaniu nesprávnych typov motorov.
\end{itemize}

\subsection{Týždeň 8}
Dlho očakávaný príchod motorov Faulhaber so zabudovanými enkódermi.
\begin{itemize}
    \item \textbf{Stretnutie:} 09/04/2024
    \item \textbf{Cieľ: } Spojazdnenie enkóderov a návrh 3D modelov potrebných pre motory Faulhaber.
    \item \textbf{Zoznam Úloh:}
\end{itemize}

\begin{table}[H]
    \centering
    \begin{tabular}{|l|p{0.6\linewidth}|}
        \hline
        \textbf{Člen} & \textbf{Úloha} \\
        \hline
        Eduard Zelenay &  Tvorba UDP servera a aplikácie pre vizualizáciu dát\\
        \hline
        Erik Môcik & Tvorba UDP servera a aplikácie pre vizualizáciu dát \\
        \hline
        Filip Lobpreis & Regulátor pre nové motory \\
        \hline
        Jakub Močarník & Dopasovanie rozmerov komponentov \\
        \hline
        Maroš Kocúr & Opravenie DPS na vyššie napätie enkóderov \\
        \hline
        Matúš Machata & Nová tlač kolies a prevodovky z dôvodu nedostatočnej presnosti tlačiarne, diskusia ohľadom nahradenia prevodovky vyrobenou vstrekolisom \\
        \hline
        Timotej Polc & Revízia kódu, ladenie gyroskopu \\
        \hline
    \end{tabular}
\end{table}
\newpage
\subsection{Týždeň 9}
Finálna verzia myšky.
\begin{itemize}
    \item \textbf{Stretnutie:} 16/04/2024
    \item \textbf{Cieľ: } Testovanie všetkých senzorov a ovládačov.
    \item \textbf{Zoznam Úloh:}
\end{itemize}

\begin{table}[H]
    \centering
    \begin{tabular}{|l|p{0.6\linewidth}|}
        \hline
        \textbf{Člen} & \textbf{Úloha} \\
        \hline
        Eduard Zelenay & Tvorba aplikácie pre vizualizáciu dát \\
        \hline
        Erik Môcik & Tvorba aplikácie pre vizualizáciu dát \\
        \hline
        Filip Lobpreis & Implementácia algoritmov \\
        \hline
        Maroš Kocúr & Synchronizovanie jadier na MCU \\
        \hline
        Timotej Polc & Implementácia algoritmov \\
        \hline
    \end{tabular}
\end{table}

\subsection{Týždeň 10}
Synchronizovanie dát s UDP serverom a regulátorom.
\begin{itemize}
    \item \textbf{Stretnutie:} 23/03/2024
    \item \textbf{Cieľ: } Implementácia algoritmov.
    \item \textbf{Zoznam Úloh:}
\end{itemize}

\begin{table}[H]
    \centering
    \begin{tabular}{|l|p{0.6\linewidth}|}
        \hline
        \textbf{Člen} & \textbf{Úloha} \\
        \hline
        Eduard Zelenay & Písanie dokumentácie \\
        \hline
        Erik Môcik & Písanie dokumentácie \\
        \hline
        Filip Lobpreis & Testovanie algoritmov \\
        \hline
        Maroš Kocúr & Technologická podpora tímu \\
        \hline
        Timotej Polc & Testovanie algoritmov \\
        \hline
    \end{tabular}
\end{table}
\newpage
\subsection{Týždeň 11}
Nájdenie správnych parametrov regulátorov a odfiltrovanie chybných dát zo snímačov.
\begin{itemize}
    \item \textbf{Stretnutie:} 30/04/2024
    \item \textbf{Cieľ: } Testovanie algoritmov na reálnom zariadení.
    \item \textbf{Zoznam Úloh:}
\end{itemize}

\begin{table}[H]
    \centering
    \begin{tabular}{|l|p{0.6\linewidth}|}
        \hline
        \textbf{Člen} & \textbf{Úloha} \\
        \hline
        Eduard Zelenay & Tvorba bludiska na testovanie \\
        \hline
        Erik Môcik & Tvorba bludiska na testovanie \\
        \hline
        Filip Lobpreis & Ladenie algoritmov \\
        \hline
        Jakub Močarník & Tvorba bludiska na testovanie \\
        \hline
        Maroš Kocúr & Technologická podpora tímu \\
        \hline
        Matúš Machata & Tvorba bludiska na testovanie \\
        \hline
        Timotej Polc & Ladenie algoritmov \\
        \hline
    \end{tabular}
\end{table}

\subsection{Týždeň 12}
Myška zvláda prechádzať širšie koridory.
\begin{itemize}
    \item \textbf{Stretnutie:} 07/05/2024
    \item \textbf{Cieľ:} Optimalizácia myšky na prejdenie súťažných rozmerov bludiska.
    \item \textbf{Zoznam Úloh:}
\end{itemize}

\begin{table}[H]
    \centering
    \begin{tabular}{|l|p{0.6\linewidth}|}
        \hline
        \textbf{Člen} & \textbf{Úloha} \\
        \hline
        Eduard Zelenay & Pomoc pri práci s Git \\
        \hline
        Erik Môcik & Finálizovanie dokumentácie a riadenia projektu \\
        \hline
        Filip Lobpreis & Optimalizácia algoritmov a logiky \\
        \hline
        Maroš Kocúr & Optimalizácia algoritmov a logiky \\
        \hline
        Timotej Polc & Optimalizácia algoritmov a logiky \\
        \hline
    \end{tabular}
\end{table}
